\chapter{GTUThesis Class} \label{ch:class}

\texttt{GTUThesis.cls} (see \Cref{sec:cls}) is the star of the show here, where the style of the thesis of GTU is defined. The user is expected to use some functions, and the demo \texttt{main.tex} shows how it is used. But for the sake of completion,  here is the documentation for the functions which the user is expected to call in their main.

\section{Declare Class}\label{sec:declare}

	\texttt{~~~~\textbackslash documentclass[lang,degree]\{GTUThesis\}}

The \texttt{lang} and \texttt{degree} options are set to set the language of the thesis (including titles and predefined text), and the degree which mainly changes some slight things in the Frontmatter. \texttt{lang} can take either \texttt{en} or \texttt{tr} with the former being the default, and \texttt{degree} takes either \texttt{undergrad} or \texttt{graduate} with the latter being the default.

\section{GTU Fields}

These are the fields required to be filled in at the beginning of the documents, they all take one argument in the following matter

	\texttt{~~~~\textbackslash GTUField\{argument\}}

and they are the following.

\begin{table}
    \caption{GTU-fields and their arguments}
    \label{tab:fields}
    \centering
    \begin{tabular}{|l|l|l|}
        \hline
        \textbf{GTU Field} & \textbf{Taken Argument} \\
        \hline
        GTUAuthor & Name of the author of the thesis (student)\\
        GTUTitle & The title of the thesis\\
        GTUFaculty & The faculty or institute of the author\\
        GTUDepartment & The department of the author\\
        GTUProject & The project the author is working on (ex. PhD thesis)\\
        GTUSupervisor & Name of the supervisor\\
        GTUYear & The year of the publication of the thesis\\
        GTUJury & Names of the jury for the thesis (comma separated) \\
        GTUDefenceDate & The date when the author presents the project to the jury\\
        GTUDecreeNo & The decree number of the jury formation (only graduate)\\
        GTUDecreeDate & The above decree's date (only graduate)\\
        \hline
    \end{tabular}
\end{table}


\section{GTU Make}

These are two functions which produce the front-matter and the back-matter of the thesis.

	\texttt{\textbackslash GTUMakeFront}

The function above produces the front-matter including the cover, the lists of content, figures, tables, and acronyms etc. in the correct order and in the correct format for the declared class's arguments (see \Cref{sec:declare} for more).

	\texttt{\textbackslash GTUMakeBack\{arg\}}

The function above produces the back-matter including the bibliography, and the optional CV and appendices. \texttt{arg} is a string that takes in the optional sections which the user wants to add comma separated (\texttt{cv} for the CV and \texttt{ap} for the appendices). If the user still doesn't want any optional sections, they still have to add the empty \texttt{\{\}} since the function waits for an argument which can be an empty string. So, the valid uses of the function are 

	\texttt{\textbackslash GTUMakeBack\{\} \% for only bibliography}
	
	\texttt{\textbackslash GTUMakeBack\{cv\} \% for bibliography and CV}
	
	\texttt{\textbackslash GTUMakeBack\{ap\} \% for bibliography and appendices}
	
	\texttt{\textbackslash GTUMakeBack\{cv,ap\} \% for bibliography, CV, and appendices}

\section{A Section to catch some other things}


Let $X_1, X_2, \ldots, X_n$ be a sequence of independent and identically distributed random variables with $\text{E}[X_i] = \mu$ and $\text{Var}[X_i] = \sigma^2 < \infty$, and let
\begin{equation}
    S_n = \frac{X_1 + X_2 + \cdots + X_n}{n}
      = \frac{1}{n}\sum_{i}^{n} X_i
\end{equation}
denote their mean. Then as $n$ approaches infinity, the random variables $\sqrt{n}(S_n - \mu)$ converge in distribution to a normal $\mathcal{N}(0, \sigma^2)$.

\begin{quote}
    \lipsum[10] \footnote{The text in this quote was created with \texttt{lipsum} package}
\end{quote}